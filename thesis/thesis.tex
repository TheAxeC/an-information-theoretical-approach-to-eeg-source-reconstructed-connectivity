\documentclass[master=mai, masteroption=ecs]{kulemt} 
% \documentclass[master=ecws, masteroption=ai]{kulemt}
\setup{title={An Information Theoretical Approach to EEG Source-Reconstructed Connectivity},
  author={Axel Faes},
  promotor={Prof.\,dr.\,ir.\ Marc Van Hulle},
  assessor={Mansoureh Fahimi\\Prof.\,dr. Daniele Marinazzo},
  assistant={Mansoureh Fahimi}}
% The following \setup may be removed entirely if no filing card is wanted
\setup{filingcard,
  translatedtitle=,
  udc=681.3*I20,
  shortabstract={
Determining how distinct brain regions are connected and communicate with each other will shed light on how behaviour emerges. In EEG studies, interpreting connectivity measures can be problematic, due to the high correlation between signals recorded from the scalp surface, a result of the volume conductance of the scalp and skin. Therefore, meaningful connectivity patterns can be measured only from the from the spatiotemporal distribution of localised cortical sources, generally referred to as source reconstruction. Still, spurious connectivity issues may persist in source reconstructed EEG data, rendering it vital to choose an appropriate measure of connectivity.
This thesis takes an information theoretical approach, which concerns model-free, probability based methods such as Conditional Mutual Information, Directed Information, and Directed feature information. We will investigate how these measures are affected by volume conduction, using as ground truth connectivity between simulated cortical sources in the brainstorm toolbox. In order to validate our methods further, these tools will also be compared with their statistical counterparts such as partial correlation, granger causality and dynamic causal modelling.
The student will start by studying state-of-the-art literature concerning source localisation and the problem of volume conduction. The student will also familiarise himself with information theoretical measures of brain connectivity. Afterwards, these measures will be applied to high density EEG datasets provided by the lab of computational neuroscience, but also to simulated source activity as a validation. The novelty lies in the usage of these information theoretical algorithms for source-reconstructed activity.
}}

% Choose the main text font (e.g., Latin Modern)
\setup{font=lm}

%% Some recommended packages.
\usepackage{booktabs}   %% For formal tables:
                        %% http://ctan.org/pkg/booktabs
\usepackage{subcaption} %% For complex figures with subfigures/subcaptions
                        %% http://ctan.org/pkg/subcaption

\usepackage{array}
\usepackage{mathpartir}
\usepackage{xspace}
\usepackage{stmaryrd}
\usepackage{amsmath}
\usepackage{listings}
% \usepackage{newtxmath}
\usepackage{float}
\usepackage{pdfpages}
\usepackage{bm}

\usepackage{todonotes}

%% Tikz Needed packages
\usepackage{pgfplots}
\pgfplotsset{width=7cm,compat=1.8}
\usepackage{pgfplotstable}
\renewcommand*{\familydefault}{\sfdefault}

% Finally the hyperref package is used for pdf files.
% This can be commented out for printed versions.
\usepackage[pdfusetitle,colorlinks,plainpages=false]{hyperref}

% \input{tex/00-macros}

\lstset{
  columns=fullflexible,
  frame=single,
  breaklines=true,
  postbreak=\mbox{\textcolor{red}{$\hookrightarrow$}\space},
}

\begin{document}

\renewcommand{\prefacename}{Acknowledgements}
\begin{preface}
  This thesis has been quite a long journey. I was able to get invested into an interesting field of research and learn a lot of new things. Both about research, and about myself. \\ 
\\
  I would like to thank everybody who kept me busy and supported me the last year. I would also like to thank the jury for reading the text. 
\end{preface}

\tableofcontents*

\setlength{\parindent}{0pt}
\setlength{\parskip}{1em}

\begin{abstract}
Determining how distinct brain regions are connected and communicate with each other will shed light on how behaviour emerges. In EEG studies, interpreting connectivity measures can be problematic, due to the high correlation between signals recorded from the scalp surface, a result of the volume conductance of the scalp and skin. Therefore, meaningful connectivity patterns can be measured only from the from the spatiotemporal distribution of localised cortical sources, generally referred to as source reconstruction. Still, spurious connectivity issues may persist in source reconstructed EEG data, rendering it vital to choose an appropriate measure of connectivity.

This thesis takes an information theoretical approach, which concerns model-free, probability based methods such as Conditional Mutual Information, Directed Information, and Directed feature information. We will investigate how these measures are affected by volume conduction, using as ground truth connectivity between simulated cortical sources in the brainstorm toolbox. In order to validate our methods further, these tools will also be compared with their statistical counterparts such as partial correlation, granger causality and dynamic causal modelling.
  
The student will start by studying state-of-the-art literature concerning source localisation and the problem of volume conduction. The student will also familiarise himself with information theoretical measures of brain connectivity. Afterwards, these measures will be applied to high density EEG datasets provided by the lab of computational neuroscience, but also to simulated source activity as a validation. The novelty lies in the usage of these information theoretical algorithms for source-reconstructed activity.
\end{abstract}

\listoffigures

% Now comes the main text
\mainmatter

\chapter{Introduction}
Within the field of computational neuroscience, extensive research is being done on the connectivity of brain regions. Understanding how distinct brain regions are connected can answer many different questions. It will shed light on how behaviour emerges. 

There are multiple ways to study brain connectivity. Brain connectivity can be studied on different levels, such as microscale or macroscale. Another question is what kind of connectivity is being measured. There is a distinction between the structural (or anatomical) connectivity and the functional connectivity. At the macroscale, functional connectivity can be measured with multiple imaging techniques. One popular imaging technique is EEG.

In EEG studies, interpreting connectivity measures can be problematic, due to the high correlation between signals recorded from the scalp surface, a result of the volume conductance of the scalp and skin. Therefore, meaningful connectivity patterns can be measured only from the from the spatiotemporal distribution of localised cortical sources, generally referred to as source reconstruction. Still, spurious connectivity issues may persist in source reconstructed EEG data, rendering it vital to choose an appropriate measure of connectivity.

This thesis takes an information theoretical approach, which concerns model-free, probability based methods such as Conditional Mutual Information and Directed Information. These methods are applied on real data, provided by the lab of computational neuroscience. This dataset contains source-reconstructed EEG data, recorded in an experiment revolving around semantic processing. 

The experiment focussed on two word categories: abstract words and concrete words. The data contains several brain regions of interest. The semantic processing of abstract and concrete words is slightly different. Some of the semantic processing is done in separate brain regions. There also is a region where semantic processing occurs in both the abstract and concrete experiments, in this thesis, this is called the common region.

\section{Research Questions}

From the motivation, there are two big aspects that this thesis deals with. There is information theory and source-reconstructed EEG Data. The novelty of this thesis lies in the usage of these information theoretical algorithms for source-reconstructed EEG data.

\begin{itemize}
\item How can information theoretical measures be applied to source-reconstructed EEG Data?
\item Which observations can be made by applying these information theoretical measures on a high density EEG dataset provided by the lab of computational neuroscience?
\end{itemize}

\section{Approach}

The goal of this thesis, as well as it's main contribution, is to apply information theoretical measures on a high density EEG dataset provided by the lab of computational neuroscience. There are two steps to accomplish this. First, an implementation has to be made that can perform an information theoretical analysis of source-reconstruted EEG Data. Secondly, the information theoretical measures are applied to a dataset. Important in this step is to decide how to analyse the data.

\begin{enumerate}
\item Study of the relevant state-of-the-art literature and theoretical background. This includes source reconstruction, brain connectivity and information theory.
\item Development of framework for information theoretical analysis of source-reconstruted EEG Data.
\item Application of information theoretical measures to high density EEG dataset provided by the lab of computational neuroscience.
\end{enumerate}

\section{Results}

The main novelty lies in the application of information theoretical algorithms for source-reconstructed EEG data. Behind this novelty, there are several different deliverables that made it possible to reach the goal of this thesis. 

The implementation of the information theoretical algorithms is a first result of this thesis. The implementation is the backbone behind the thesis. The implementation consists out of three parts. There is the data conversion, which converts the source-reconstructed EEG data into multiple format. Secondly, methods for calculating the correct amount of bins to discretize the continious data have been implemented. 

Finally, the actual information theoretical equations have been implemented. The implementation is developed to be of general use, meaning that it isn't too difficult to convert the implementation into an open source connectivity package.

A second result of this thesis is the actual analysis of the high density EEG dataset provided by the lab of computational neuroscience. The main conclusion made in the analysis of the EEG dataset, is that, within the common region, the actual semantic processing is quite different between the semantic processing of abstract words and the semantic processing of concrete words.

\section{Structure of the Thesis}

Chapter~\ref{source-reconstruction} discusses source reconstruction. It starts by explaining what source reconstruction entails and why it is important. Afterwards, an explanation is given about the reasons that spurious connectivity issues may still persist in source reconstructed EEG data.

Chapter~\ref{connectivity} gives a background of brain connectivity and volume conduction. this chapter hightlights the different kinds of connectivity and brievely discusses them.

Chapter~\ref{information-theory} provides a background about information theory. This chapter starts by explaining the concept of information entropy. Using this concept, the different aspects from information theory are discussed, including joint entropy, mutual information, multivariate methods and directed information. Finally, this chapter details methods on how to use information theory on continuous data.

Chapter~\ref{eeg-experiment} details the experiment that generated the data which this thesis analyses. This chapter starts by detailing the reasons behind the experiment and then details the experiment itself.

Chapter~\ref{evaluation} will go in depth into the actual analysis that has been done on the data. The binning and the different analysis experiments are described. 

Chapter~\ref{implementation} goes in depth into the source code that has been developed for the analysis. There are several important aspects to be discussed about the source code, such as the programming language and the implementation of the equations.

Chapter~\ref{related} brievely discusses some related work. More specifically, granger causality is detailed. Chapter~\ref{future-work} details several interesting directions we can take this research. Considering the conclusion of this thesis, that information theoretical algorithms work well on source-reconstructed EEG data, many new paths open for further investigation. Finally, chapter~\ref{conclusion} concludes the thesis.

\chapter{Source Reconstruction}
Source reconstruction refers to the localisation of the electrical activity of the brain. In the case of this thesis, we are referring to EEG source localization, which is also called the reverse problem. Electrical activity within the brain is measured on the scalp using EEG electrodes. This is the forward problem. The reverse problem maps the measurements from the EEG electrodes back into the brain. \cite{schoffelen2009source}

\section{Reverse Problem}

Figure~\ref{source-local-img} visualizes the reverse problem. This is an ill-posed problem. EEG electrodes measure the scalp, in effect measuring a 2D grid. However, the brain is a 3D object. This means that information is inherently lost when EEG is used. One of the consequences is there are many different solutions to the reverse problem. 

\begin{figure}[!htb]
\caption{EEG Source Localisation \cite{source-loc-img}.}
\label{source-local-img}
    \centering
    \includegraphics[width=\textwidth]{fig/source_recon}
\end{figure}

Equation~\ref{reverse} represents source localisation. $X$ represents the scalp recorded EEG activity. $S$ represents the electrical sources within the brain, a current density vector. $L$ represents the head volume conductor model. The reverse problem is about finding $S$, this is represented in equation~\ref{reverse2}.

\begin{equation}\label{reverse}
X = LS + n
\end{equation}

\begin{equation}\label{reverse2}
O(S) = min||X - LS||^2 
\end{equation}

\section{Head Volume Conductor Model}

Several different head volume conductor models can be used. The two most popular are the simple head models and the realistic head models. Simple head models model the brain as a single sphere with a couple layers. This model also assumes a uniform medium within the brain. Using a simple head model is fast and simple. However, it is not accurate. 

The realistic head model is an accurate model, but it is computationally much more expensive. Different techniques are used to construct a head model such as finite element or finite boundary techniques. 

\section{Dipoles}

The electrical activity from neural populations can be represented as a dipole. The inverse problem is essentially about determining the location, orientation, amplitude and number of these dipoles

There are practical limits to the number of dipoles that can be used. One of the most important limits is the spatial resolution of EEG. EEG can only measure electrical activity with a certain accuracy. This is due to the fact that the electrical activity is measured on the scalp, and each electrode records a linear combination of the underlying neural sources. This happens due to the low conductivity of the layers that are separating the cortex from the surface of the scalp (such as the skull, skin, and cerebrospinal fluid). The phenomena of how different mixtures of sources reach the scalp, how they attenuate and spread through the different layers of the head is referred to as volume conduction. We will elaborate on this further in the thesis.

\section{Algorithms}\label{source-alg}

There are different kind of algorithms used to solve the inverse problem. There is dipole fitting, which involves only a small number of dipoles. The locations, orientations, and magnitudes of these dipoles needs to be calculated. For some experiments, this method can be good enough, since a single dipole can account for 80\% of all electrical activity \cite{cohen2014analyzing}. Dipole fitting can be performed by a number of different software programs or matlab toolboxes, including BESA, eeglab and fieldtrip \cite{hoechstetter2004besa, delorme2004eeglab, oostenveld2011fieldtrip}.

Then there are nonadaptive distributed-source imaging methods and adaptive distributed-source imaging methods. The idea of distributed-source imaging is that thousands of dipoles are placed within the brain on fixed locations with fixed orientations. This leaves only the magnitude to be computed. This is done by assigning electrode weights to each dipole.

Nonadaptive methods compute these electrode weights based on the electrode locations. This means that the weights are fixed over time and frequency. sLORETA is a commonly used nonadaptive inverse-source imaging technique \cite{pascual2002standardized}. The provided data used in this thesis has also been source-reconstructed using sLORETA. Other nonadaptive methods include LORETA and minimum-norm estimator \cite{pascual1994low, hamalainen1994interpreting}.

Nonadaptive methods are relatively fast to compute, are applicable to single time points and their result looks like FMRI activation maps. However, there are also some disadvantages. One issue is the number of comparisons that need to be computed in statistical analyses. In the case of 10,000 dipoles over time and frequency with two experiment conditions, there over more than 100 million possible statistical comparisons that need to be computed. \cite{cohen2014analyzing}

Adaptive distributed-source imaging have a different way of computing the electrode weights. The recorded data is also used to compute the weights. The weights are not fixed over time and frequency. The accuracy of adaptive methods is often quite high, but they are much more complicated with many more parameters to be set. Beamforming is a class of algorithms that is most commonly used adaptive distributed-source imaging method \cite{gross2001dynamic, van1997localization}. 

\section{Practical Limits}

Within a simulated environment, high spatial localization accuracy can be obtained. However, in practive, there are many issues. There are always uncertainties regarding electrode positions, brain anatomy, head movement and scalp conductivity. This means that spatial accuracy is often a few centimeters in size. The smallest voxels which are created by source reconstruction are typically 5-10 mm$^3$ in size. \cite{grech2008review} 

Without knowledge of the inverse problem, it might seem that source reconstruction is not a big deal. It might even seem to only bring advantages, since you can work within the actual cortical areas. However, source reconstruction comes with its fair share of problems and inaccuracies. 

Within the context of this thesis, knowledge of the inverse problem is essential. While the data was already source-reconstructed, for the reasons above, it is important to know the inverse problem. 

\chapter{Brain Connectivity}
The human brain is anatomically a conglomeration of different brain regions. Functionally, the human brain is a giant network of specialised units connected by dynamically configurable communication pathways. \cite{faes2012methodological}

There are several reasons why scientists would want to measure the way the brain is connected. Understanding brain connectivity aids our knowledge of the working of the brain. It allows us to better understand sensorimotor and cognitive tasks that are performed. This can lead to improved diagnosing of various diseases such as aphasia. \cite{horwitz2003elusive}

A connectivity analysis refers to any analysis where multiple signals are utilised at the same time. The signal can represents several different sources. The signals could be the recordings from different electrodes or, in the case of this thesis, the source-reconstructed EEG data. 

\section{Connectivity Measures}

Different connectivity measures can highlight different aspects about the connectivity. This means that there is not one superior connectivity method. The different connectivity measures are: \cite{friston2011functional, cohen2014analyzing}

\begin{itemize}
\item Phase-Based Connectivity
\item Power-Based Connectivity
\item Cross-Frequency Coupling
\item Graph Theory
\item Granger Causality
\item Information Theory
\end{itemize}

\subsection{Phase-based Connectivity}

Phase-based connectivity analyses utilise the phase differences between different signals. This is a very popular connectivity measure and this is partly because it has a strong neurophysiological interpretation. The timing of neural populations, as measured through phase, become synchronized. This method is computationally fast, does not make many assumptions on the data and is insensitive to time lag. However, it relies on precise temporal relationships.

\subsection{Power-based Connectivity}

Power-Based Connectivity analyses utlise time-frequency power correlations. These correlations are computed accross time or over trials. Power-Based Connectivity analyses are fairly resistant against temporal noise.

\subsection{Cross-Frequency Coupling}

Cross-Frequency Coupling is a statistical relationship between signals in different frequency bands. When utilised on electrodes, it can infer local connectivity when measured at a single electrode. With multiple electrodes, longer-range connectivity can be inferred. One advantage is that findings can be linked across species. Cross-frequency coupling is also very strong at identifying task-related high-frequency power. Computationally speaking, there are some disadvantages to cross-frequency coupling. The search space is huge, which makes it computationally very expensive.

\subsection{Graph Theory}

Graph Theory is a mathematical framework for studying graphs. A graph, containing vertices and edges, can be used to model networks. These networks can be models for brain connectivity. When electrode signals are used, each node can represent an electrode and connectivity is represented by edges. Graph-theoretical analyses are often easy to interpret. Graph theory presents a generic framework to analyse networks, making it easy to apply the same analyses to different kinds of data. The main task becomes the construction of a network. Graph theory is relatively new in the context of computational neuroscience. This makes graph theory an interesting research direction, but this is also a disadvantage. Since graph theory within computational neuroscience isn't as well established, it can be difficult to compare findings to different studies.

\subsection{Granger Causality}

Granger Causality is a statistical method that utilises signal variance. It tests whether variance from one signal can predict variance in another signal at a certain point in time. One big advantage to granger causality is that it can find directional connectivity. It can ignore simultaneous connectivity. This makes it less susceptible to volume conduction. Granger causality is a computationally expensive method to perform. \cite{bressler2011wiener}

\subsection{Information Theory}

Information theory is the main focus of this thesis. It is a simple, yet flexible and robust method for computing connectivity. One useful aspect from information theory is mutual information. Mutual information computes shared information between two variables. This computation is based on the (joint) distributions of values within variables. Information theory has several advantages. It can detect multiple kinds of relationships between different signals. It can detect both linear and non-linear relationships. Information theory also contains many different constructs which are useful for a connectivity analysis. Mutual information is one of them. Another is multivariate generalisations of mutual information, which can work with more than two signals. Channel-coding theory can be used for signal transmission integrity. 

Mutual information is, by far, the most popular method from information theory. One disadvantage is that mutual information does not provide any information on what kind of relationship there is between variables. Information theory is generally defined for discrete distributions, with different techniques to discretize continuous data. However, the method and the chosen parameters for discretizing continuous data have a big effect on the computed results. Information theory is generally also a computationally expensive operation to perform. Finally, there are no clear neurophysiological interpretations for information theoretical constructions. Information theory is discussed much more in-depth in the following chapter, chapter~\ref{information-theory}.

\section{Connectivity Interpretation}

There are several key aspects to keep in mind when interpreting results from a connectivity analysis. Different connectivity measures may or may not take these aspects into consideration. In order to provide a proper connectivity analysis, these key aspects and the properties of the connectivity measures need to be known.

Typically, two different signal are not fully in sync. When an analysis is performed on the signal from multiple electrodes, there may be a phase lag between these electrodes. Most connectivity measures do not take this phase lag into consideration. This does not mean that that these connectivity measures cannot be used, as long as the phase lag is consistent. \cite{horwitz2003elusive}

Figure~\ref{timelag} shows a second key aspect. In this example, region C is connected with A and B. Because C$\rightarrow$A is faster than C$\rightarrow$B, it may appear that there is a phase lag, even though there is no causal or direct relation between A and B.

\begin{figure}[!htb]
\caption{Time Lag \cite{cohen2014analyzing}.}
\label{timelag}
    \centering
    \includegraphics[width=\textwidth]{fig/timelag}
\end{figure}

\section{Volume Conduction}

The head volume conductor model, discussed in chapter~\ref{source-reconstruction}, is quite interesting. The brain conducts electrical activity, this is how the electrical activity can be measured on the scalp. Volume conduction refers to this process of conducting electrical activity through a medium. \cite{brunner2016volume}

Figure~\ref{conduction} shows several problems that the reverse problem has to deal with. In situation A, there would be no problem. In this situation, every EEG electrode measures one and only one electrical source within the brain. However, this is not what happens in reality. 

\begin{figure}[!htb]
\caption{Volume Conduction \cite{cohen2014analyzing}.}
\label{conduction}
    \centering
    \includegraphics[width=\textwidth]{fig/conduction}
\end{figure}

Reality is a combination of situations B and C \cite{brunner2016volume}. Situation B shows that electrical sources in the brain generate large electromagnetic fields which are recorded by more than one EEG electrode. Situation C shows that the scalp also conducts electricity. These two situations have a big effect on connectivity measures. 


\chapter{Information Theory}\label{information-theory}
% \section{Information Theory}
Information theory provides us the tools to study the information processing capabilities of different systems. These systems include computers, artificial intelligence and also the brain. The story of Information Theory begins with Shannon who provided the tools to estimate information \cite{shin1949mathematical}. The most fundamental aspect of information theory is the concept of the 'bit'. 

As a standard, information theory deals with 'bits'. One bit of information represents a choice between two equally probable options. A perfectly balanced coin toss contains one bit of information. It has 50\% chance of landing on heads and 50\% chance of landing on tails. A bit is a measure of information and a measure of uncertainty. Thus uncertainty and information are tightly intertwined. If we are completely certain about a certain event, there is no information to be gained. 

This chapter will cover the most important aspects of information theory. First and foremost, this means discussing entropy. Once we have an understanding of entropy, extensions can be discussed. These include joint entropy and conditional entropy. With these tools, we can go into the area of mutual information, which plays an important role in this thesis. Another important aspect is the generalisation into multivariate systems and dealing with continuous (as opposed to discrete) systems. Finally, some explanation is provided of coding theory and cybernetics.

\section{Entropy}

The next step is entropy. There are different kinds of entropy, such as thermodynamic entropy. Here, we are talking about information entropy. Information entropy is the average uncertainty associated with a random variable. In other words, information entropy is the average rate of information from a random or stochastic variable. 

In order to calculate the entropy, we require some discrete random variable X, with ${x_1 ... x_n}$ the different values from X. $P(x)$ is the probability mass function. The entropy $H(X)$ can be calculated as:

\begin{equation}
H(X) = -\sum_{i=1}^{n}P(x_i)log_2(P(x_i))
\end{equation}

Interesting to note is that $log_2(P(x_i))$ is the information about value $x_i$. This formula can be extended into the conditional entropy of two events $X$ and $Y$. The entropy $H(X|Y)$ is the entropy of random variable X given that the outcome of Y is known.

\begin{equation}
H(X|Y) = \sum_{i,j}P(x_i, y_j)log_2(\frac{P(y_j)}{P(x_i, y_j)}) = -\sum_{i,j}P(x_i, y_j)log_2(\frac{P(x_i, y_j)}{P(y_j)})
\end{equation}

If random variables X and Y are independent of each other, then $H(X|Y) = H(X)$. If the variables are completely independent, knowing anything about Y, will not change anything we know about X. Similarly, if $H(X|Y) = 0$, then X is completely determined by Y. 

The rule of Bayes is also applicable to conditional entropy:

\begin{equation}
H(X|Y) = H(Y|X) + H(X) - H(Y)
\end{equation}

\section{Joint Entropy}

$H(X)$ and $H(X|Y)$ are basic notions of information measures. Figure~\ref{entropy} visualises the notion of entropy. This figure also contains two information measures that are not yet described, $I(X,Y)$ and $H(X,Y)$. Respectively, they are the mutual information and the joint entropy.

\begin{figure}[!htb]
\caption{Venn diagram for information measures.}
\label{entropy}
    \centering
    \includegraphics[width=0.8\textwidth]{fig/entropy}
\end{figure}

In the figure, $H(X,Y)$ is the complete information content of X and Y. Using the following equation, we can calculate $H(X,Y)$:

\begin{equation}
H(X,Y) = -\sum_{i=1}^{n}\sum_{j=1}^{m}P(x_i, y_j)log_2(P(x_i, y_j))
\end{equation}

Using figure~\ref{entropy}, we can see some interesting relations. $H(X,Y)$ is related to $H(X|Y)$ and $H(X)$:
\begin{equation}\label{joint}
H(X,Y) = H(X|Y) + H(Y)
\end{equation}

This can be validated:
\begin{align*}
H(X,Y) &= -\sum_{i=1}^{n}\sum_{j=1}^{m}P(x_i, y_j)log_2(P(x_i, y_j))\\
&= -\sum_{i=1}^{n}\sum_{j=1}^{m}P(x_i, y_j)log_2(P(y_i)P(x_i | y_j))\\
&= -\sum_{i=1}^{n}\sum_{j=1}^{m}P(x_i, y_j)log_2(P(y_i))-\sum_{i=1}^{n}\sum_{j=1}^{m}P(x_i, y_j)log_2(P(x_i | y_j))\\
&= -\sum_{j=1}^{m}P(y_j)log_2(P(y_i))-\sum_{i=1}^{n}\sum_{j=1}^{m}P(x_i, y_j)log_2(P(x_i | y_j))\\
&= H(Y) + H(X|Y)
\end{align*}

This relation is useful for the actual implementation of information theoretical algorithms. An important note about entropy, conditional entropy and joint entropy is that they are non-negative. It would not make sense for a random variable to have a negative information content. Joint entropy is also always greater or equal to individual entropy and joint entropy is smaller or equal to the sum of individual entropies.

\begin{equation}
H(X) \ge 0
\end{equation}

\begin{equation}
H(X|Y) \ge 0
\end{equation}

\begin{equation}
H(X,Y) \ge 0
\end{equation}

\begin{equation}
H(X,Y) \ge H(X)
\end{equation}

\begin{equation}
H(X,Y) \le H(X) + H(Y) 
\end{equation}

\section{Mutual Information}
Mutual information is the amount of information that is common between two variables. Figure~\ref{entropy} shows the mutual information as the intersection between $H(X)$ and $H(Y)$. This can also be seen in figure~\label{mutual}. The figure also shows how mutual information can be computed. The individual entropies are summed and the joint entropy is subtracted. Another equal interpretation of mutual information is that mutual information measures the reduction in uncertainty or information when we observe one of the variables.

\begin{figure}[!htb]
\caption{Venn diagram for mutual information.}
\label{mutual}
    \centering
    \includegraphics[width=0.8\textwidth]{fig/mutual}
\end{figure}

\begin{equation}\label{info:mutual}
I(X,Y) = H(X) + H(Y) - H(X,Y)
\end{equation}

\begin{equation}
I(X,Y) = H(X) - H(X|Y)
\end{equation}

Mutual information can also be calculated using the probability mass functions directly. In this case we get:

\begin{equation}
I(X,Y) = \sum_{i=1}^{n}\sum_{j=1}^{m}P(x_i, y_j)log_2(\frac{P(x_i, y_j)}{P(x_i)P(y_j)})
\end{equation}
    
While mutual information does show whether there is a relationship or correlation between two variables, it does not give information about the "shape" of the relationship. Making the analogy with figure~\ref{entropy}, mutual information shows how much overlap there is, but it does not explain anything else. Mutual information makes no assumptions of what the distribution of the variables X and Y is like. 

Mutual information, like entropy, is non-negative. Additionally, just like there is conditional entropy, there is also conditional mutual information. Figure~\ref{mutual} shows the conditional mutual information. It refers to the entropy in common between $X$ and $Y$, with the exclusion of $Z$. Conditional mutual information indicates the mutual information given another variable is given:

\begin{equation}
I(X,Y|Z) = \sum_{k=1}^{o}\sum_{i=1}^{n}\sum_{j=1}^{m}P(x_i, y_j, z_k)log_2(\frac{P(x_i, y_j z_k)P(z_k)}{P(x_i, z_k)P(y_j z_k)})
\end{equation}

\section{Multi-Variate Information Theory}\label{multivariate}
In the previous sections, information theory has been observed through a bivariate lens. These methods can be generalised to multivariate variations. When comparing different brain regions, we do not want to restrict ourselves to a bivariate case.

Joint entropy can be easily extended to a multivariate case. In the bivariate case, the joint probability mass function was used and a summation over both random variables was done. The multivariate case simply generalises this equation:

\begin{equation}\label{multivariateentropy}
H(X_1, ..., X_n) = -\sum_{x_1}...\sum_{x_n}P(x_1,...,x_n)log_2(P(x_1,...,x_n))
\end{equation}

Equation~\ref{joint} can also be extended into a multivariate case. In this case, we get:

\begin{equation}
H(X_1, ..., X_n) = -\sum_{k=1}^{n}H(X_k|X_{k-1},...,X_1)
\end{equation}

The multivariate case of conditional entropy becomes:

\begin{equation}\label{condentropy2}
H(Y | X_1, ..., X_n) = H(Y, X_1, ..., X_n) - H(X_1, ..., X_n)
\end{equation}

We can also make a formula to calculate the entropy of multiple variables conditioned on a single variable:

\begin{equation}\label{condentropy}
H(X_1, ..., X_n | Y) = H(Y, X_1, ..., X_n) - H(Y)
\end{equation}

Having extended the different forms of entropy into a multivariate case, we can make a multivariate method for mutual information. The multivariate mutual information becomes a recursive function:

\begin{equation}
I(X_1, ..., X_{n+1}) = I(X_1, ..., X_n) - I(X_1, ..., X_n | X_{n+1})
\end{equation}

The multivariate mutual information can also be decomposed into a som of entropies, which makes it easier to calculate:

\begin{equation}\label{mut}
I(X_1, ..., X_n) = \sum_{T \subseteq \{1,...,n\}} (-1)^{|T|}H(T)
\end{equation}

\begin{equation}\label{condmut}
I(X_1, ..., X_n | Y) = \sum_{T \subseteq \{1,...,n\}} (-1)^{|T|}H(T|Y)
\end{equation}
    
With these formulas, the discussed entropies are formulated in a multivariate way. This specific formulation of multivariate mutual information is also called interaction information in the literature.

Another interesting note is that most equations can be reduced into a sum of (joint) entropies. Equation~\ref{condmut} is formulated as a sum of conditional entropies. But with equation~\ref{condentropy}, we can reduce equation~\ref{condmut} into a sum of entropies without conditional variables. 

Being able to reduce most equations into a sum of entropies becomes a useful ability during the implementation of these equations.

\section{Directed Information}\label{directed-information}

Regular mutual information shows the amount of entropy that is shared between two or more random variables. The mutual information can capture both linear and non-linear relationships. However, mutual information cannot show the information flow between different variables. 

In order to show the information flow, an extension to mutual information has to be made. This extension is called directed information. Directed information uses processes rather than variables. A process $X^n$ is a sequence, a vector of data $X_1,...,X_n$. A time series, by its nature, can be seen as a process.

\begin{equation}
I(X^n \rightarrow Y^n) = \sum^{n}_{i=1}I(X^i, Y_i | Y^{i-1})
\end{equation}

An additional note is that directed information is computationally expensive to compute. Multivariate mutual information has to be calculated $n$ times. This makes computation of long processes slow to compute. 

\section{Continuous Data}\label{binning}
Up until this point, the discussed equations assumed that discrete data was available. However, within the context of neuroscience and the brain, data is almost always continuous in nature. In the case of EEG research, the data comprises of time series of electrical activity.

In the case of this thesis, the data is a time series of source-reconstructed EEG data. In order to compute entropy with continuous data, the data needs to be discretized. There are multiple ways to discretize data:

\begin{itemize}
\item Histogram analysis
\item Correlation analysis
\item Clustering analysis
\item Decision tree analysis
\item Probability modelling
\item Binning
\end{itemize}

These ways utilise different tools, such as histograms, correlation, clustering, decision trees, probability distributions, in order to discretize the data. 

The default choice is to utilise binning. Binning is a relatively simple process that is fast to compute. This is what makes binning attractive. Utilisation of other ways to discretize data is left for future work.

\subsection{Binning}

Binning is the process of putting the continuous data into a predetermined number of bins. In order to avoid adding any kind of noise or bias to the data, every bin should have an equal width. In order to map the continuous data onto the bins, every bin needs to represent a certain interval. The interval of the first bin starts at the minimum value of the continues data. The interval of the last bin ends at the maximum value of the continuous data. The width of the bin is decided by the number of bins. 

\begin{figure}[!htb]
\caption{Binning representation.}
\label{entropy}
    \centering
    \includegraphics[width=0.5\textwidth]{fig/binning}
\end{figure}

The big question is how many bins should be used to discretize the data. If too little bins are used, information from the continuous data is lost. On the other hand, with a very large amount of bins, each bins will only contain one value. In this case, generality is lost. 

The number of bins used has a large impact on the entropy, so making the correct choice is important. There are multiple ways to make the correct choice. The number of bins can be estimated empirically by plotting the entropy with varying bin sizes. Using this method, the issues that come with a too small or too large bin size can be vizualised. 

There is another, more statistically sound, method for determining the appropriate amount of bins. There is the Freedman-Diaconis rule. This rule states that the optimal amount of bins is related to the interquartile range $Q_x$, the amount of data point $n$ and the minimum $min(x)$ and maximum $max(x)$ elements of the data:

\begin{equation}\label{binequation}
nbins = \frac{max(x) - min(x)}{2Q_xn^{-1/3}}
\end{equation}

\section{Summary}

This chapter provides the necessary mathematical background in order to accomplish the information theoretical analysis of the EEG source-reconstructed data. As explained in section~\ref{multivariate}, most equations can be reduced into a sum of (joint) entropies, which becomes important for the implementation. 

To summarize, the two most important aspects from this chapter are:

\begin{enumerate}
\item Multivariate mutual information
\item Dealing with continuous data
\end{enumerate}

In order to accomplish the analysis from this thesis, the continuous data has to be discretized. More specifically, this thesis uses the binning method in order to discretize the data. The analysis itself focusses on multivariate mutual information. Chapter~\ref{evaluation} will go in depth into the actual analysis that has been done on the data. The binning and the different analysis experiments are described. Chapter~\ref{implementation} goes in depth into the source code that has been developed for the analysis. There are several important aspects to be discussed about the source code. 

\chapter{EEG Experiment}
Applying information theoretical algorithms on a high density EEG dataset is the main focus of this thesis. The EEG dataset is provided by the lab of computational neuroscience of KU Leuven. In order to perform an analysis of a dataset, it is important to understand what kind of data we are dealing with.

This chapter discusses the experiment used to generate the dataset. The experiment and the reasons behind the experiment are introduced. Afterward the actual experiment is discussed. This starts with the an explanation of the EEG experiment itself and the paradigm. Afterwards the source localization is discussed. Finally, the final structure of the data is brievely discussed.

\section{Introduction}

The experiment resolves around the human brain's representation of semantic categories. By using high density EEG recordings, the semantic processing of words within the human brain was captured. Within the literature, there are several different theories and concepts that explain the representation of semantic categories within the human brain. The human brain is a complicated organ with many different cortical areas. Each theory focusses on different cortical areas and explain how they are involved with the representation of semantic categories. These theories have been further developed with the aid of different neuroimaging experiments. 

One of the most prominant theories of semantic word processing is the grounded cognition model. The model is also called the embodied cognition model. This model explains that semantic knowledge is kept within high-level perception and motor representation systems in the brain. 

This means that a word is comprehended based on modality-specific neural systems. In other words, elements such as visual and auditory features are used to define words. 

\todo{to be rewrited}

Examples in favour of the embodied cognition theorem are fMRI studies showing that animate objects cluster in the more lateral aspect of the fusiform gyrus, whereas activations associated with inanimate or man-made objects cluster in the more medial aspect of the fusiform gyrus (Chao et al., 2015), additionally there are certain category mappings based on specific shape, such as "face" and "body" patches (Tsao et al., 2008). There is also support coming from lesion studies where damage to the brain's modal system creates category-specific deficits, disproportionately preserving categories such as animals, foods, or artefacts (Barsalou et al., 2003; Caramazza and Mahon, 2003). To explain these deficits according to the grounded cognition model, they claim natural objects such as animal, fruits and vegetables etc, are distinguished primarily from their visual semantic properties, while man-made items such as tools and vehicles are vehicles are distinguished primarily by their function, as evidenced by fMRI and PET imaging studies (Devlin et al., 2002).

The grounded cognition model, and its extension into the hubs and spoke model as an attempt to explain the ability of between category and within category differentiation despite overlapping modality-specific features (Ralph and Ralph, 2013)  can be criticized from several aspects. First of all, modality-specific features of a concept are much less variant between different subjects when shown as a clear visual stimuli (image). However, when stimuli are conceptual in nature (such as words presented in written or spoken form) the perceptual and motor-sensory function they evoke are much more difficult to control, as the experience designated to specific entities is subject-dependent to a larger degree (Kemmerer, 2015). Secondly, and especially relevant for our study is that the grounded cognition model only explains features that are rooted in physical experience with a concept. However, abstract concepts, such as "democracy" or concepts related to emotions are not explained by this theory, even though the vast majority of human language actually constitutes abstract concepts.

The difference in the processing of abstract and concrete words, has been tackled by some theories, the main ones being the dual coding theorem, and the context availability theorem (Kounios and Holcomb, 1994; Wang et al., 2010). In the dual coding theorem, the claim is that two separate systems reside in the brain, a nonverbal "imagery" system the verbal ("linguistic") system which implements the modality-specific aspects of a concepts (not unlike the grounded cognition model), and a purely verbal "linguistic" system which is more involved in the abstract form of language. According to the context availability theorem, the processing of concrete and abstract words never happens in isolation, but is aided by the context by which the word can be understood. Since the context of concrete words are constrained by their physical referents, understanding them will again involve modality-specific brain systems as defined in the grounded cognition model. In the case of abstract concepts however, context is more variable and experience dependent. 

Both theories have neither been proved nor disproved by scientific literature, and research is still in progress. One most vital limitation of all aforementioned theories is that they have been predominantly studied using functional neuroimaging techniques such as fMRI and PET, which is limited in terms of temporal resolution (Bookheimer, 2002), even though visual word recognition is a complex dynamic process that involves several cognitive stages happening on millisecond scale, such as visual encoding, lexical activation, and semantic presentation (Bentin, 1999). Whether temporal activation of these stages are sequential or partially parallel, interactive processes is not known, however it is clear that heamodynamic measures cannot follow the progress of activation of different areas, which result in a different pattern of activity for each word (Bentin, 1999; Salmelin et al., 1998). On the other hand studies involving lesions and electrical stimulations do no give a good spatial overview on the large-scale networks involved.

By virtue of its excellent temporal resolution, the EEG technique has been hailed to probe the brain's detailed processing of objects and words. Studies using EEG/ERP recordings have successfully distinguished differences in word categories on different stages of semantic processing, starting from difference in word length and word frequency revealed earlier in the time course between 100 and 200 ms (Hauk et al., 2006), to distinction in processing of semantic categories, such as abstract concepts versus concrete concepts (Bentin, 1999), animals verus tools (Simanova et al., 2010) and more generally natural objects versus artefacts (Kiefer, 2001) which occurs later in the time course as revealed by the N400 potential.

The main drawback of the EEG is it's relatively low spatial resolution, and in this way falls short in detecting differences in cortical network activation. For example, it is possible that different aspects of semantic activity and language comprehension are associated with different negativities during the same time epoch, explaining the differences in the N400 scalp distribution as being caused by different neural generators (Bentin, 1999). Therefore, spatial resolution is equally vital to obtain a complete picture of what is happening.

In this study, using high density EEG recordings, we were able to capture the fast dynamics involved in semantic processing as we localized the neural activity on the cortex with an accuracy in the range of millimetres and milliseconds. This will provide us with a unique opportunity to investigate the spreading of activity during the processing of semantic features.

\section{Materials and methods}

The experiment was done in a sound-attenuated, darkened room with a constant temperature of 20 degrees, sitting in front of an LCD screen at a distance of about 70cm.

EEG data is recorded using 128 active Ag/AgCl electrodes (SynampsRT, Compumedics, France), according to the international 10-20 system. Two of these electrodes serve as ground (AFz) and reference (FCz). The EEG signal is recorded at a 2 KHz sampling rate and downsampled to 500 Hz. All electrodes are mounted in an electrode cap that is placed on the subject's head (Easycap, Germany). 

\section{Experimental Paradigm}

In order to ensure that our subjects are involved in semantic processing and not just in lexical access (as would be the case with a lexical decision task), we choose a categorization paradigm consisting of 600 Dutch words (only nouns) taken from the database of concreteness ratings for 30,000 Dutch words (Brysbaert et al., 2014). Abstract words were selected based on having a concreteness rating of maximum 2.5, and concrete words had to have minimum of 3.5. Concreteness ratings of abstract and concrete words were found to be statistically different using t-test. Word length and word frequency for both groups of words were controlled, no significant difference  where found between. Word length and frequency were deduced from the Dutch CLEARPOND software (Marian et al., 2012). Additionally, words were pseudo randomly organized in such a way that no two consecutive words would have a high forward association, and forward associations were controlled for all consecutive words. Forward associations were taken from the Dutch free association network created by De Deyne et al (De Deyne et al., 2008). 

In each trial, a single white word is shown on a black screen for 300ms, followed by a question mark that lasts for 1 second. Prior to each trial, a fixation cross appeared on the screen, cueing subjects to focus in the middle part of the screen. Participants are asked to press the mouse button, on the moment they see the question mark, only if the word they are shown is a colour. The colour category is therefore used a non-target (called "filler") and will not be included in the results. After pressing a mouse button they received visual feedback on the button press ("kleur!" (colour) if pressed correctly, and "fout!" (wrong) otherwise). 

\begin{figure}[!htb]
\caption{Binning representation.}
\label{experiment}
    \centering
    \includegraphics[width=\textwidth]{fig/experiment}
\end{figure}

\section{Source localization}
For our source reconstruction analysis we used the Brainstorm toolbox (Tadel et al., 2011), freely available under the GNU general public license. The default anatomy was based on the ICBM-152 template. For the forward model we used OpenMEEG BEM (Gramfort et al., 2010), in which case the cortex was divided into 15,000 dipoles. Noise covariance and data covariance matrices were obtained by merging the matrices calculated from the baseline of all selected trials. As our inverse modelling method, we used sLORETA (Pascual-Marqui, 2002) as it has shown to yield zero localization error in the absence of noise and to support the reconstruction of multiple sources. Source orientation was constrained to be orthogonal to the cortical surface. The signal-to-noise ratio (SNR) was set to the default value suggested by the Brainstorm Toolbox (SNR = 3). In addition, sulci are not taken into consideration during our analysis, as accurate source localization in these regions is implausible, as stipulated in Brainstorm's documentation.

In order to verify the correctness of our procedure, we attempted to reproduce the results using different source localization algorithms, as is recommended by (Mahjoory et al., 2017). We did not pursue the entire statistical procedure, however, we did analyze our initial results by taking the average over all trials regardless of semantic features, using the four methods available in the brainstorm toolbox: wMNE, dSPM, sLORETA, and unconstrained sLORETA. Manual inspection of the results with these methods revealed that the spatial distributions are similar over time albeit with different degrees of spatial smoothing.

\section{ROI selection}

As mentioned, we used sLORETA for source map normalization. Normalizing the current density maps with respect to a reference level (estimated in our case, from pre-stimulus baseline) provides a statistical map, which is essential the signal to noise ratio of the current estimate as a function of location. Therefore we use the normalized source maps to evaluate the significance of the data and select our ROIs. For both paradigms, we averaged the sLORETA statistical maps over all trials, and eliminated activity below 75\% of the maximum activity. Afterwards we overlapped the maps for both paradigms and selected regions that were commonly active, and had an estimated square size of larger than 3cm2. This resulted in four regions on the inferior temporal gyrus, temporal pole, inferior frontal, and anterior orbital gyrus.

\begin{figure}[!htb]
\caption{Visualization of the Selected Brain Regions.}
\label{experiment-2}
    \centering
    \includegraphics[width=\textwidth]{fig/experiment-2}
\end{figure}


\chapter{Evaluation}\label{evaluation}
\section{Calculating Bin Sizes}

\begin{figure}[!htb]
\caption{Entropy comparison between bin sizes}
\label{bins}
    \centering
    \includegraphics[width=1.2\textwidth]{fig/bins}
\end{figure}

\section{Comparing Common Regions}

\begin{figure}[!htb]
\caption{Experiment}
\label{all-channel-1}
    \centering
    \includegraphics[width=\textwidth]{fig/all-channel-1}
\end{figure}

\begin{figure}[!htb]
\caption{Comparison}
\label{comp-1}
    \centering
    \includegraphics[width=\textwidth]{fig/comp-1}
\end{figure}

\section{Per Subject Comparison}
\subsection{Effect of Trials Used}

\begin{figure}[!htb]
\caption{Entropy comparison between bin sizes}
\label{10-trials}
    \centering
    \includegraphics[width=\textwidth]{fig/subject1_10trials_all-channel-1}
\end{figure}

\begin{figure}[!htb]
\caption{Entropy comparison between bin sizes}
\label{40-trials}
    \centering
    \includegraphics[width=\textwidth]{fig/subject1_40trials_all-channel-1}
\end{figure}

\begin{figure}[!htb]
\caption{Entropy comparison between bin sizes}
\label{all-trials}
    \centering
    \includegraphics[width=\textwidth]{fig/subject1_alltrials_all-channel-1}
\end{figure}

\chapter{Implementation}\label{implementation}
\section{Language}

\section{Data Conversion}

\chapter{Related Work}\label{related}
This thesis specifically focussed on information theoretical measures for brain connectivity. However, there are several other methods which can be used to measure brain connectivity.

One method especially of interest is granger causality. 

\section{Granger Causality}



\chapter{Future Work}
\section{Directed Information}

\section{Open Source Connectivity Package}

\section{Data Flow Connectivity}

\chapter{Conclusion}\label{conclusion}
The main goal of this thesis is to apply information theoretical measures of brain connectivity to a high density EEG source-reconstructed dataset. Information theoretical algorithms are becoming more and more utilised within the field of computational neuroscience. 

Information theoretical algorithms are model-free and probability based. They make no assumptions on what kind of distribution the data has. This makes information theoretical algorithms very versatile and an excellent tool to measure connectivity.

This thesis mainly focussed on 2 different information measurements. The first measurement is bivariate mutual information. The second measurement is a generalisation of mutual information, called multivariate mutual information or interaction information.

The provided EEG dataset was produced by an experiment resolving around semantic processing. More specifically, the experiment compared the brain activity between the semantic processing of concrete words and abstract words. The provided EEG dataset had already been source-reconstructed.

One cortical area of interest was an area that was active during both semantic processing of abstract and concrete words. Analysis showed that the actual activity in this area differed between the semantic processing of abstract and concrete words.

The implementation has been carefully developed such that the source code can be used in open-source projects. In the future, we want to further develop the implementation so it can be released as an open-source connectivity package which can be used by other scientists. 

In conclusion, an analysis of a high density EEG dataset has been computed and an implementation has been developed. The implementation provides a platform for further information theoretical analysis and research.  

% \appendixpage*
% \appendix

\backmatter

% The bibliography comes after the appendices.
% You can replace the standard "abbrv" bibliography style by another one.
\bibliographystyle{abbrv}
\bibliography{bib/main}
\nocite{*}

\end{document}
