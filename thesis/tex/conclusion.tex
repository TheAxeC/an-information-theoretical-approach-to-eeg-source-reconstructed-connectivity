The main goal of this thesis is to apply information theoretical measures of brain connectivity to a high density EEG source-reconstructed dataset. Information theoretical algorithms are becoming more and more utilised within the field of computational neuroscience. 

Information theoretical algorithms are model-free and probability based. They make no assumptions on what kind of distribution the data has. This makes information theoretical algorithms very versatile and an excellent tool to measure connectivity.

This thesis mainly focussed on 2 different information measurements. The first measurement is bivariate mutual information. The second measurement is a generalisation of mutual information, called multivariate mutual information or interaction information.

The provided EEG dataset was produced by an experiment resolving around semantic processing. More specifically, the experiment compared the brain activity between the semantic processing of concrete words and abstract words. The provided EEG dataset had already been source-reconstructed.

One cortical area of interest was an area that was active during both semantic processing of abstract and concrete words. Analysis showed that the actual activity in this area differed between the semantic processing of abstract and concrete words.

The implementation has been carefully developed such that the source code can be used in open-source projects. In the future, we want to further develop the implementation so it can be released as an open-source connectivity package which can be used by other scientists. 

In conclusion, an analysis of a high density EEG dataset has been computed and an implementation has been developed. The implementation provides a platform for further information theoretical analysis and research.  